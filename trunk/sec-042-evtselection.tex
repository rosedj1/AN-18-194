\section{Event selection}
\label{sec:evtselection}

\subsection{Baseline selection}
Events considered in this analysis are first required to fire various HLT 
paths as described in Section~\ref{sec:trigpaths}. Duplicate events in 
each primary datasets are removed accordingly. Events are also required to 
have at least one good primary vertices. No MET filters are applied as 
the \met variable is not used in the main analysis and hence the analysis 
is not sensitive to noise effects or beam backgrounds.

\subsection{ZZ candidate selection}
\label{sec:zzcandsel}

The four-lepton candidates are built from {\bf selected leptons}, which  
are the tight leptons 
%(defined in sections~\ref{sec:eleID} and~\ref{sec:muonReco}) 
that pass the ${\rm SIP_{3D}} < 4$ vertex constraint
and the isolation cuts 
%(defined in sections~\ref{sec:eleiso} and~\ref{sec:muoniso}), 
where FSR photons are subtracted as described in~\cite{AN-16-442,AN-17-342}
[Check if this sentence is true for 2017 and 2018].
A lepton cross cleaning is applied 
by discarding electrons which are within $\Delta R < 0.05$ of selected muons. 

The construction and selection of four-lepton candidates proceeds 
according to the following sequence:
\begin{enumerate}
\item {\bf Z candidates} are defined as pairs of selected leptons
 of opposite charge and matching flavour ($e^+ e^-$, \, $\mu^+\mu^-$)
 that satisfy $4 < m_{\ell\ell(\gamma)} < 120~\GeVcc$, where the Z candidate mass
 includes the selected FSR photons if any.
\item {\bf ZZ candidates} are defined as pairs of non-overlapping Z candidates.
 The Z candidate with reconstructed mass $m_{\ell\ell}$ closest to the nominal Z boson
 mass is denoted as ${\rm Z_1}$, and the second one is denoted as ${\rm Z_2}$.
 ZZ candidates are required to satisfy the following list of requirements:
  \begin{itemize} 
  \item {\bf Ghost removal }: $\Delta R(\eta,\phi) > 0.02$ between each of the four leptons.
  \item {\bf lepton $p_T$}: Two of the four selected leptons should pass 
     $p_{T,i} > 20~\GeVc$ and $p_{T,j} > 10~\GeVc$.
  \item {\bf QCD suppression}: all four opposite-sign pairs that can
     be built with the four leptons (regardless of lepton flavor)
     must satisfy $m_{\ell\ell} > 4~\GeVcc$.
     Here, selected FSR photons are not used in computing $m_{\ell\ell}$, 
     since a QCD-induced low mass dilepton (eg. $J/\Psi$) 
     may have photons nearby (e.g. from $\pi_0$). 
  \item {\bf ${\rm Z_1}$ mass}: $m_{\rm Z1} > 40~\GeVcc$
  \item {\bf 'smart cut'}: defining ${\rm Z_a}$ and ${\rm Z_b}$ as 
     the mass-sorted alternative pairing Z candidates 
     (${\rm Z_a}$ being the one closest to the nominal Z boson mass),
     require NOT($|m_{\rm Za}-m_{\rm Z}| < |m_{\rm Z1}-m_{\rm Z}|$ AND $m_{\rm Zb}<12$).
     Selected FSR photons are included in $m_{\rm Z}$'s computations.
     This cut discards $4\mu$ and $4e$ candidates where the alternative pairing
     looks like an on-shell Z + low-mass $\ell^+ \ell^-$. 
  %\item {\bf four-lepton invariant mass}: $\mllll > 70~\GeVcc$
  \item {\bf four-lepton invariant mass}: $118~\GeV < \mass{4\ell} > 130~\GeVcc$
  \end{itemize}	
\item Events containing at least one selected ZZ candidate form the {\bf signal region}.
\end{enumerate}	

If more than one ZZ candidate survives the above selection, the ZZ candidate with the ${\rm Z_1}$ closest in 
mass to nominal Z boson mass is chosen.
