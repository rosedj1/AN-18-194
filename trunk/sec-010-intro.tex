\section{Introduction}

Following the discovery of the Higgs boson with the ATLAS and CMS experiment~\cite{paper:Aad:2012,paper:Chatrchyan:2012}, 
a thorough program of precise measurements and searches has 
been carried out on this particle, both to uncover possible deviations from the Standard Model (SM) or decipher the nature of the Higgs 
sector. In particular, various exotic Higgs decays have been widely considered, in which small deviations in the Higgs decay 
width or discovery of exotic decay modes could mean hints of Beyond-Standard-Model (BSM) physics.

One well-motivated class of BSM models that could lead to exotic Higgs decays includes theories with a hidden "dark" 
sector~\cite{Curtin:2014cca,Curtin:2013fra,Davoudiasl:2013aya,Davoudiasl:2012ag,Gopalakrishna:2008dv}.
Among various BSM final states presented in these theories,  
the focus here is on the process $p p \rightarrow h \rightarrow Z \zd \rightarrow 4\ell$, with \zd 
representing a new vector boson in the dark sector. 
%Section~\ref{sec:datamc} introduces the signal model considered in this analysis 
%and Section~\ref{sec:app-01-sigmc} describes the validation of kinematic properties of this signal model to the state-of-the-art 
%event generator used by various $p p \rightarrow h \rightarrow Z Z^* \rightarrow 4\ell$ analysis.
This signal process is kinematically similar to the SM process $p p \rightarrow h \rightarrow Z Z^* \rightarrow 4\ell$, 
therefore similar analysis techniques can be employed to perform this search. This search uses $pp$ collision data at a centre-of-mass energy 
$\sqrts = 13~\TeV$ collected by the CMS detctor, using the full Run 2 dataset with \usedLumi. \zd is assumed to decay to electron 
and muon, giving rise to the $4e$, $4\mu$, $2e2\mu$, $2\mu2e$ final state. Assuming only on-shell decay, only the mass range 
$\mass{\zd} < 35~\GeV$ is kinematically possible. In this analysis, a mass range of $4~\GeV < \mass{\zd} < 35~\GeV$ is considered. 
The narrow mass window around the meson $\Upsilon$ state (\mass{\Upsilon}) is excluded.

The outline of this document is as follows: An introduction to the dark photon model is described in Section~\ref{sec:model}. 
Section~\ref{sec:datamc} introduces data and MC samples used in this analysis.
Section~\ref{sec:obj} describes definitions of various physics objects. Section~\ref{sec:strat}-\ref{sec:bkgd} details 
the analysis strategy and method. Section~\ref{sec:syst} describes various sources of systematic unceratinties. 
Section~\ref{sec:yield} presents various yields and distributions after signal region selections, which are used for 
inputs for results and interpretations. Section~\ref{sec:result} introduces the likelihood model and presents interpretations 
with the HAHM model, in particular, upper limits on the kinematic mixing parameter $\epsilon$ and $Br(H \rightarrow Z \zd)$ 
are reported.
