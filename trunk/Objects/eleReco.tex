More details on electron reconstruction can be found in Ref.~\cite{ElectronLegacy}. 

Electron candidates are preselected using loose cuts on track-cluster matching observables, so as to preserve the highest possible efficiency while rejecting part of the QCD background. To be considered for the analysis, electrons are required to have a
transverse momentum $p^e_T >$ 7 GeV, a reconstructed $|\eta^e| <$ 2.5, and to satisfy a loose primary vertex 
constraint defined as $d_{xy} < 0.5$ and $d_z < 1$. 
Such electrons are called {\bf loose electrons}.

The early runs in the 2016 data-taking exhibit an tracking inefficiency originating from a reduced hit reconstruction efficiency in the strip detector (``HIP" effect). 
The resulting data-MC discrepancy is corrected using scale factors as is done for the electron selection with data efficiencies measured using the same tag-and-probe technique outlined later (see Section~\ref{sec:eleEffMeas}). 
These studies are carried out by the EGM POG and the results are only summarised here.

The electron reconstruction scale factors are shown Fig.~\ref{fig:ele_rec_scale_factors} as a one-dimensional function of the super cluster $\eta$ only, as it was shown that the \pt dependence of the scale factor is negligible.

\begin{figure}[!htb]
\vspace*{0.3cm}
\begin{center}
\includegraphics[width=0.6\textwidth]{Figures/Electrons/ele_rec_scale_factors.pdf}
\end{center}
\caption{Electron reconstruction efficiencies efficiency in data versus $\eta$ and data/MC scale factors as provided by the EGM POG.}
\label{fig:ele_rec_scale_factors}
\end{figure}
