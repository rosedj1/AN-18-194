Vector Boson Fusion (VBF) and other production mechanisms of Higgs Boson normally differ as regards the jet kinematics. 
In this analysis, jets are thus used for the event categorization, which will be introduced in Section~\ref{sec:categorization}.

\subsubsection{Jet Identification}

Jets are reconstructed by using the anti-$k_T$ clustering algorithm out of particle flow candidates, with a distance parameter $R = 0.4$, 
after rejecting the charged hadrons that are associated to a pileup primary  vertex.

To reduce instrumental background, the loose working point jet ID suggested by the JetMET Physics Object Group is applied. 
In this analysis, the jets are required to be within $|\eta| < 4.7$ area and have a transverse momentum above 30 GeV. 
In addition, the jets are cleaned from any of the tight leptons (passing the SIP and isolation cut computed after FSR correction) 
and FSR photons by a separation criterion: $\Delta R(\text{jet,lepton/photon}) > 0.4$.


\subsubsection{Jet Energy Corrections}

The calorimeter response to particles is not linear
and it is not straightforward to translate the measured jet energy
to the true particle or parton energy, therefore we need Jet Energy Corrections.
In this analysis, standard jet energy corrections are applied to the reconstructed jets,
which consist of L1 Pileup, L2 Relative Jet Correction,
L3 Absolute Jet Correction for both Monte Carlo samples and data,
and also residual calibration for data.

% Figure~\ref{fig:jets} shows the comparisoin between data and MC for the leading jet in Z events with exactly one jet,
% where a selection $\Delta\phi(Z,{\rm jet})>2.5$ has been applied.

% \begin{figure}[!h]
% \centering
% \includegraphics[width=0.49\linewidth]{Figures/Jets/Histo_etaj1_2e_dataeff.pdf}
% \includegraphics[width=0.49\linewidth]{Figures/Jets/Histo_etaj1_2mu_dataeff.pdf}
% \caption{Comparison between data and MC for jet $\eta$ in Z + 1 jet events. \label{fig:jets}}
% \end{figure}


\subsubsection{B-tagging}

For categorization purpose, we need to distinguish whether a jet is b-jet or not.
The \emph{Combined Secondary Vertex} algorithm is used as our b-tagging algorithm.
It combines information about impact parameter significance,
the secondary vertex and jet kinematics.
The variables are combined using a likelihood ratio technique to compute the b-tag discriminator.
In this analysis, a jet is considered to be b-tagged if it passes the \emph{CSVv2M} working point,
i.e. if its \verb|pfCombinedInclusiveSecondaryVertexV2BJetTags| discriminator is greater than 0.8484~\cite{btagReferenceEffsRun2}.

Data to simulation scale factors for b-tagging efficiency are provided for this working point for the full dataset as a function of jet $\pt$, $\eta$ and flavour.
They are applied to simulated jets by downgrading (upgrading) the b-tagging status of a fraction of the b-tagged (untagged) jets that have a scale factor smaller (larger) than one.
