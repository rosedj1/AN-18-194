The relative isolation for electrons is then defined as: 

\begin{equation}
\text{RelPFiso} = (\sum_{\text{charged}} p_T + \sum^{\text{corr}}_{\text{neutral}} p_T)/p_T^{\text{lepton}}  .
\label{eqn:elepfrelisoeqn}
\end{equation} 

where the corrected neutral component of isolation is then computed using the formula :

\begin{equation}
\label{eqn:neutralea}
  \sum^{\text{corr}}_{\text{neutral}} p_T = \text{max}(\sum^{\text{uncorr}}_{\text{neutral}} p_T - \rho \times A_\text{eff},0 \GeV)  .
\end{equation}

and the mean pile-up contribution to the isolation cone is obtained as :  
\begin{equation}
 PU =  \rho \times A_\text{eff}
\label{eqn:purho}
\end{equation}

where $\rho$ is the mean energy density in the event and the effective area $A_{eff}$ is defined as the ratio
between the slope of the average isolation and that of $\rho$ as a function of the number of vertices. 

The electron isolation working point was optimized in Ref.~\cite{AN-15-277} and the electron isolation working was 
chosen to be $\text{RelPFiso}(\Delta R = 0.3) < 0.35$. 
