Electrons in data are corrected for features in ECAL energy scale
in bins of $\pt$ and $\left| \eta \right|$. Corrections are calculated
on a $\cPZ \to \Pe\Pe$ sample to align the dielectron 
mass spectrum in the data to that in the MC, and to
minimize its width.

The $\cPZ \to \Pe\Pe$ mass resolution in Monte Carlo is made to match
data by applying a pseudorandom Gaussian smearing to electron energies,
with Gaussian parameters varying in bins of $\pt$ and $\left| \eta \right|$.
This has the effect of convoluting the electron energy spectrum with a
Gaussian.



The electron energy scale is measured in data by fitting a Crystall-ball function to the di-electron mass spectrum around the Z peak in the $Z+\ell$ control region. 
The energy scale for the full 2016 dataset is shown in Fig.~\ref{fig:ele_energy_scale}(a) and agrees with the MC with 100~MeV. 
The stability of the energy scale across different run periods is shown in Fig.~\ref{fig:ele_energy_scale}(b), where the data is binned into approximately 500~pb luminosity blocks.

\begin{figure}[!htb]
\vspace*{0.3cm}
\begin{center}
\subfigure [] {\resizebox{7.5cm}{!}{\includegraphics{Figures/Electrons/ele_energy_scale.pdf}}}
\subfigure [] {\resizebox{9.5cm}{!}{\includegraphics{Figures/Electrons/ele_energy_scale_per_lumi.pdf}}}
\end{center}
\caption{
(a): electron energy scale measured in the $Z+\ell$ control region for EB and EE electrons. The results of the Crystall-ball fit are reported in the figure. 
(b): lepton energy scales per 500~pb luminosity block. 

}
\label{fig:ele_energy_scale}
\end{figure}


%\begin{figure}[!htb]
%\vspace*{0.3cm}
%\begin{center}
%\includegraphics[width=0.7\textwidth]{Figures/Electrons/ele_energy_scale}
%\end{center}
%\caption{Electron energy scale measured in the $Z+\ell$ control region. The results of the Crystall-bal fit are reported in the figure.}
%\label{fig:ele_energy_scale}
%\end{figure}
%
%\begin{figure}[!htb]
%\vspace*{0.3cm}
%\begin{center}
%\includegraphics[width=0.8\textwidth]{Figures/Electrons/ele_energy_scale_per_lumi}
%\end{center}
%\caption{Lepton energy scales per 500~pb luminosity block. Note: discrepancy in run G was fixed (bug in calibration application); need to update plot.}
%\label{fig:ele_energy_scale_per_lumi}
%\end{figure}

%The effect of these corrections on a single-$\cPZ$ sample is shown in
%Fig.~\ref{fig:ZeeCorrections} for, from left to right, with both electrons in the barrel,
%one electron in the endcap and one in the barrel, and both electrons in the endcap. The top
%row shows the fit to the data and MC distributions before the correction, and the bottom
%row shows the fits after the correction.

%\begin{figure}[htbp]
%  \begin{center}
%    \subfigure[]{\includegraphics[width=0.32\textwidth]{Figures/Zee-BB_comp_before.pdf}}
%    \subfigure[]{\includegraphics[width=0.32\textwidth]{Figures/Zee-BE_comp_before.pdf}}
%    \subfigure[]{\includegraphics[width=0.32\textwidth]{Figures/Zee-EE_comp_before.pdf}} \\
%    \subfigure[]{\includegraphics[width=0.32\textwidth]{Figures/Zee-BB_comp_after.pdf}}
%    \subfigure[]{\includegraphics[width=0.32\textwidth]{Figures/Zee-BE_comp_after.pdf}}
%    \subfigure[]{\includegraphics[width=0.32\textwidth]{Figures/Zee-EE_comp_after.pdf}} \\
%    \caption{
%      Effect of electron energy corrections and smearing for single-$\cPZ$
%      events with both electrons in the barrel (left), one electron in the
%      barrel and one in the endcaps (middle), and both electrons in the
%      endcaps (right). Plots in the top row are uncorrected, and plots in the
%      bottom row have the corrections and smearing applied.
%    }
%    \label{fig:ZeeCorrections}
%  \end{center}
%\end{figure}
