

The FSR recovery algorithm was considerably simplified with respect to what was done in Run I, while maintaining similar performance. 
The selection of FSR photons is now only done per-lepton and no longer depends on any Z mass criterion, thus much simplifying the subsequent ZZ candidate building and selection. As regards the association of photons with leptons, the rectangular cuts on $\Delta R(\gamma,l)$ and $E_{T,\gamma}$  have been replaced by a cut on $\Delta R(\gamma,l)/E_{T,\gamma}^{2}$.

Starting from the collection of 'PF photons' provided by the particle-flow algorithm, the selection of photons and their association to a lepton proceeds as follows:
\begin{enumerate}
\item The preselection of PF photons is done by requiring $p_{T,\gamma} > 2~\GeV$, $|\eta^{\gamma}| < 2.4$, and a relative Particle-flow isolation smaller than $1.8$. The latter variable is computed using a cone of radius $R=0.3$, a threshold of $0.2~\GeV$ on charged hadrons with a veto cone of $0.0001$, and $0.5~\GeV$ on neutral hadrons and photons with a veto cone of $0.01$, also including the contribution from pileup vertices (with the same radius and threshold as per charged isolation) .
\item Supercluster veto: we remove all PF photons that match with any electron passing both the loose ID and SIP cuts. The matching is peformed by directly associating the two PF candidates.
\item Photons are associated to the closest lepton in the event among all those pass both the loose ID and SIP cuts.
\item We discard photons that do not satisfy the cuts $\Delta R(\gamma,l)/E_{T,\gamma}^2 < 0.012$, and $\Delta R(\gamma,l)<0.5$.
\item If more than one photon is associated to the same lepton, the lowest-$\Delta R(\gamma,l)/E_{T,\gamma}^2$ is selected.
\item For each FSR photon that was selected, we exclude that photon from the isolation sum of all the leptons in the event that pass both the loose ID and SIP cuts. This concerns the photons that are in the isolation cone and outside the isolation veto of said leptons ($\Delta R < 0.4$ AND $\Delta R > 0.01$ for muons and $\Delta R < 0.4$ AND ($\eta^{\text{SC}} < 1.479$ OR $\Delta R > 0.08$) for electrons).
\end{enumerate}

More details on the optimization of the FSR photon selection can be found in Ref.~\cite{AN-15-277, AN-16-217}.
